%%%%%%%%%%%%%%%%%%%%%%%%%%%%%%%%%%%%%%%%%
% Journal Article
% LaTeX Template
% Version 1.4 (15/5/16)
%
% This template has been downloaded from:
% http://www.LaTeXTemplates.com
%
% Original author:
% Frits Wenneker (http://www.howtotex.com) with extensive modifications by
% Vel (vel@LaTeXTemplates.com)
%
% License:
% CC BY-NC-SA 3.0 (http://creativecommons.org/licenses/by-nc-sa/3.0/)
%
%%%%%%%%%%%%%%%%%%%%%%%%%%%%%%%%%%%%%%%%%

%----------------------------------------------------------------------------------------
%	PACKAGES AND OTHER DOCUMENT CONFIGURATIONS
%----------------------------------------------------------------------------------------

\documentclass[twoside,twocolumn]{article}
\usepackage{amsmath}
\usepackage{blindtext} % Package to generate dummy text throughout this template 

\usepackage[sc]{mathpazo} % Use the Palatino font
\usepackage[T1]{fontenc} % Use 8-bit encoding that has 256 glyphs
\linespread{1.05} % Line spacing - Palatino needs more space between lines
\usepackage{microtype} % Slightly tweak font spacing for aesthetics

\usepackage[english]{babel} % Language hyphenation and typographical rules

\usepackage[hmarginratio=1:1,top=32mm,columnsep=20pt]{geometry} % Document margins
\usepackage[hang, small,labelfont=bf,up,textfont=it,up]{caption} % Custom captions under/above floats in tables or figures
\usepackage{booktabs} % Horizontal rules in tables

\usepackage{lettrine} % The lettrine is the first enlarged letter at the beginning of the text

\usepackage{enumitem} % Customized lists
\setlist[itemize]{noitemsep} % Make itemize lists more compact

\usepackage{abstract} % Allows abstract customization
\renewcommand{\abstractnamefont}{\normalfont\bfseries} % Set the "Abstract" text to bold
\renewcommand{\abstracttextfont}{\normalfont\small\itshape} % Set the abstract itself to small italic text

\usepackage{titlesec} % Allows customization of titles
\renewcommand\thesection{\Roman{section}} % Roman numerals for the sections
\renewcommand\thesubsection{\roman{subsection}} % roman numerals for subsections
\titleformat{\section}[block]{\large\scshape\centering}{\thesection.}{1em}{} % Change the look of the section titles
\titleformat{\subsection}[block]{\large}{\thesubsection.}{1em}{} % Change the look of the section titles

\usepackage{fancyhdr} % Headers and footers
\pagestyle{fancy} % All pages have headers and footers
\fancyhead{} % Blank out the default header
\fancyfoot{} % Blank out the default footer
\fancyhead[C]{Running title $\bullet$ May 2016 $\bullet$ Vol. XXI, No. 1} % Custom header text
\fancyfoot[RO,LE]{\thepage} % Custom footer text

\usepackage{titling} % Customizing the title section

\usepackage{hyperref} % For hyperlinks in the PDF

%----------------------------------------------------------------------------------------
%	TITLE SECTION
%----------------------------------------------------------------------------------------

\setlength{\droptitle}{-4\baselineskip} % Move the title up

\pretitle{\begin{center}\Huge\bfseries} % Article title formatting
\posttitle{\end{center}} % Article title closing formatting
\title{FYS4150 - Project 1} % Article title
\author{%
\textsc{Heine H. Ness \& Sindre R. Bilden}\thanks{A thank you or further information} \\[1ex] % Your name
\normalsize University of Oslo \\ % Your institution
\normalsize \href{mailto:h.h.ness@fys.uio.no}{h.h.ness@fys.uio.no}\ ; \href{mailto:s.r.bilden@fys.uio.no}{s.r.bilden@fys.uio.no}% Your email address
%\and % Uncomment if 2 authors are required, duplicate these 4 lines if more
%\textsc{Jane Smith}\thanks{Corresponding author} \\[1ex] % Second author's name
%\normalsize University of Utah \\ % Second author's institution
%\normalsize \href{mailto:jane@smith.com}{jane@smith.com} % Second author's email address
}
\date{\today} % Leave empty to omit a date
\renewcommand{\maketitlehookd}{%
\begin{abstract}
\noindent \blindtext % Dummy abstract text - replace \blindtext with your abstract text
\end{abstract}
}

%----------------------------------------------------------------------------------------

\begin{document}

% Print the title
\maketitle

%----------------------------------------------------------------------------------------
%	ARTICLE CONTENTS
%----------------------------------------------------------------------------------------

\section{Introduction}

\lettrine[nindent=0em,lines=3]{T}his project will examinate different techniques for approximating the solution to a differential equation where a continious function is known. The equation describes an electrostatic potential $\Phi$ generated by a localized charge density $\rho(\vec{r})$ and is usualy described - in three dimentions - by:
\begin{equation}
\nabla^2\Phi = -4\pi \rho(\vec{r}) \label{eq:Poisson3D}
\end{equation}
If $\rho(\vec{r})$ is spherical symmetric, eq. \ref{eq:Poisson3D} may be written in a one-dimentional manner by substituting $\phi(r)=r\Phi(r)$:
\begin{equation}
\frac{d^2\phi(r)}{dr^2}=-4\pi r\rho(r) \label{eq:Poisson1D}
\end{equation}
By rewriting eq. \ref{eq:Poisson1D} to a general form it reads:
\begin{equation}
-u''(x)=f(x)
\end{equation}
In this spesific case, the Poisson equation is solved by \textit{Gaussian elimination} of a set of linear equations, both in a general manner and an optimized way of a spesific matrix. The optimized method is later compared with another general method called \textit{LU-decomposition}.

%------------------------------------------------

\section{Methods}
The methods used in this projects are the following:
\begin{itemize}
\item Dirichlet boundary conditions
\item Nummerical derivation
\item Gaussian elimination
\item LU-decomposition
\end{itemize}
\subsection{Dirichlet boundary condition}
Dirichlet boundary conditions - also refered to as fixed boundary condition - specifies the value of a given function on a surface $T=f(r,t)$. In a one-dimentional problem it translates to defining an interval of $x$ - $x\in [x_{min},x_{max}]$ - and the function values $f(x_{min})=f_l$ and $f(x_{max})=f_h$ at the edges of the intervall.
\subsection{Nummerical derivarion}
The derivative of a discrete funtion may be found by nummerical derivation. The principle of nummerical derivation is a result of Taylor expansion. By expanding a function from   a point $x$ with a step $h$, two equations form depending on the direction:
\begin{equation}
f(x+h) = f(x)+hf'(x)+\frac{h^2}{2}f''(x)\ldots \label{eq:taylor_pos}
\end{equation}
\begin{equation}
f(x-h) = f(x)-hf'(x)+\frac{h^2}{2}f''(x)\ldots \label{eq:taylor_neg}
\end{equation}
By adding eq. \ref{eq:taylor_neg} to eq. \ref{eq:taylor_pos}, a approximation for the second derivative is achieved.
\begin{equation}
f'' = \frac{f_+-2f+f_-}{h^2}+\frac{h^4}{6h^2}f^{\mathit{IV}}\label{eq:sec_der}
\end{equation}
Where $f_+ = f(x+h)$, $f=f(x)$, $f_-=f(x-h)$ and $f^{IV}$ is the fourth derivative of $f(x)$. By truncating the series at the fourth derivative a small mathematical error - $\mathcal{O}$ - appears in the order of $h^2$. If a discrete funtion is introduced where $f_i = f(x_i) = f(c_0+ih)$, eq. \ref{eq:sec_der} may be rewritten to an algorithm for the nummerical second derivative.
\begin{equation}
f''_i = \frac{f_{i+1}-2f_i+f_{i-1}}{h^2}\label{eq:sec_der_discrete}
\end{equation}
In eq. \ref{eq:sec_der_discrete} the mathematical error $\mathcal{O}(h^2)$ is neglected.
\subsection{Gaussian elimination}
Gaussian elimination is a method for simplifying a set of linear equations. It is easly visualized through a matrix notation of $Ax = y$ where $A$ and $y$ is known.
\begin{equation}
\begin{bmatrix}
a_11 & a_12 & a_13\\
a_21 & a_22 & a_23\\
a_31 & a_32 & a_33
\end{bmatrix} 
\begin{bmatrix}
x_1 \\ x_2 \\ x_3
\end{bmatrix}=
\begin{bmatrix}
y_1 \\ y_2 \\ y_3
\end{bmatrix}
\end{equation}
Gaussian elimination is often divided into two main parts, forward and backward substitution.
\subsubsection{Forward substitution}
The forward substitution is focusing to 
\subsection{LU-decompostition}

Text requiring further explanation\footnote{Example footnote}.

%------------------------------------------------

\section{Results}

\begin{table}[p]
\caption{Example table}
\centering
\begin{tabular}{llr}
\toprule
\multicolumn{2}{c}{Name} \\
\cmidrule(r){1-2}
First name & Last Name & Grade \\
\midrule
John & Doe & $7.5$ \\
Richard & Miles & $2$ \\
\bottomrule
\end{tabular}
\end{table}

\blindtext % Dummy text

\begin{equation}
\label{eq:emc}
e = mc^2
\end{equation}

\blindtext % Dummy text

%------------------------------------------------

\section{Discussion}

\subsection{Subsection One}

A statement requiring citation \cite{Figueredo:2009dg}.
\blindtext % Dummy text

\subsection{Subsection Two}

\blindtext % Dummy text

%----------------------------------------------------------------------------------------
%	REFERENCE LIST
%----------------------------------------------------------------------------------------

\begin{thebibliography}{99} % Bibliography - this is intentionally simple in this template

\bibitem[Figueredo and Wolf, 2009]{Figueredo:2009dg}
Figueredo, A.~J. and Wolf, P. S.~A. (2009).
\newblock Assortative pairing and life history strategy - a cross-cultural
  study.
\newblock {\em Human Nature}, 20:317--330.
 
\end{thebibliography}

%----------------------------------------------------------------------------------------

\end{document}
