\documentclass[twoside,twocolumn]{article}
\usepackage{amsmath}
\usepackage{blindtext} % Package to generate dummy text throughout this template 
\usepackage{graphicx}
\usepackage{natbib}

\usepackage[sc]{mathpazo} % Use the Palatino font
\usepackage[T1]{fontenc} % Use 8-bit encoding that has 256 glyphs
\linespread{1.05} % Line spacing - Palatino needs more space between lines
\usepackage{microtype} % Slightly tweak font spacing for aesthetics

\usepackage[english]{babel} % Language hyphenation and typographical rules

\usepackage[hmarginratio=1:1,top=32mm,columnsep=20pt]{geometry} % Document margins
\usepackage[hang, small,labelfont=bf,up,textfont=it,up]{caption} % Custom captions under/above floats in tables or figures
\usepackage{booktabs} % Horizontal rules in tables

\usepackage{lettrine} % The lettrine is the first enlarged letter at the beginning of the text

\usepackage{enumitem} % Customized lists
\setlist[itemize]{noitemsep} % Make itemize lists more compact

\usepackage{abstract} % Allows abstract customization
\renewcommand{\abstractnamefont}{\normalfont\bfseries} % Set the "Abstract" text to bold
\renewcommand{\abstracttextfont}{\normalfont\small\itshape} % Set the abstract itself to small italic text

\usepackage{titlesec} % Allows customization of titles
\renewcommand\thesection{\Roman{section}} % Roman numerals for the sections
\renewcommand\thesubsection{\roman{subsection}}
\titleformat{\section}[block]{\large\scshape\centering}{\thesection.}{1em}{} % Change the look of the section titles
\titleformat{\subsection}[block]{\large}{\thesubsection.}{1em}{} % Change the look of the section titles

\usepackage{fancyhdr} % Headers and footers
\pagestyle{fancy} % All pages have headers and footers
\fancyhead{} % Blank out the default header
\fancyfoot{} % Blank out the default footer
\fancyhead[C]{FYS4150 $\bullet$ Project 2 $\bullet$ Oktober 2016} % Custom header text
\fancyfoot[RO,LE]{\thepage} % Custom footer text

\usepackage{titling} % Customizing the title section

\usepackage{hyperref} % For hyperlinks in the PDF

%----------------------------------------------------------
%  COMMANDS
%---------------------------------------------------------

\newcommand{\nl}{

\medskip
\noindent
}
%--------------------------------------------



%----------------------------------------------------------------------------------------
%	TITLE SECTION
%----------------------------------------------------------------------------------------

\setlength{\droptitle}{-4\baselineskip} % Move the title up

\pretitle{\begin{center}\Huge\bfseries} % Article title formatting
\posttitle{\end{center}} % Article title closing formatting
\title{FYS4150 - Project 1} % Article title
\author{%
\textsc{Vegard R�nning \& Heine H. Ness \& Sindre R. Bilden} \\[1ex] % Your name
\normalsize University of Oslo \\ % Your institution
\normalsize \href{mailto:vegarduio@gmail.com}{vegarduio@gmail.com}\ ; \href{mailto:h.h.ness@fys.uio.no}{h.h.ness@fys.uio.no}\ ; \href{mailto:s.r.bilden@fys.uio.no}{s.r.bilden@fys.uio.no}\\% Your email address
\footnotesize \href{https://github.com/sindrerb/FYS4150-Collaboration/tree/master/Doc/Project2}{github.com/sindrerb/FYS4150-Collaboration/tree/master/Doc/Project2}
%\and % Uncomment if 2 authors are required, duplicate these 4 lines if more
%\textsc{Jane Smith}\thanks{Corresponding author} \\[1ex] % Second author's name
%\normalsize University of Utah \\ % Second author's institution
%\normalsize \href{mailto:jane@smith.com}{jane@smith.com} % Second author's email address
}
%----------------------------------------------------------------------------
\date{\today} % Leave empty to omit a date
\renewcommand{\maketitlehookd}{%
\begin{abstract}



\end{abstract}
}

%----------------------------------------------------------------------------

\begin{document}

% Print the title
\maketitle

%----------------------------------------------------------------------------
%	ARTICLE CONTENTS
%----------------------------------------------------------------------------

\section{Introduction}
\lettrine[nindent=0em,lines=3]{B}la bla

%----------------------------------------------------------------------------
\section{Methods}
\subsection{Jacobi's method}
Jacobi's method implements Jacobi's rotation in order to solve $\hat{A}\vec v=\lambda \vec v$. 


\subsection{Unit tests}

A unit test is a small piece of code that tests parts of a program for calculation errors. This to ensure that the program runs as expected and delivers correct results throughout the program. The unit tests we have used in this assignment are as follows.

\subsubsection{Orthogonality test}

The Jacobi method preforms orthogonal or unitary transformations to the matrix it operates on. That means the ortogonality of each column in a matrix $\hat{A}$ is conserved. 
\nl
So if $\hat{A}$ consists of orthogonal column vectors $\hat{A} = [\vec{a_1} \vec{a_2} \cdots \vec{a_n}  ]$ the dot product of any column vector can be described by a Kroniker delta.

\begin{align*}
    \vec{a_i}\cdot\vec{a_j} = \delta_{ij} =
    \begin{cases}
    1|i=j\\
    0|i\neq j
    \end{cases}
\end{align*}

For the 

%----------------------------------------------------------------------------
\section{Results and discussion}


%----------------------------------------------------------------------------
\section{Summary and Conclusion}

%----------------------------------------------------------------------------
%\twocolumn[]
\end{document}